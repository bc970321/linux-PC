% !TEX TS-program = xelatex
% !TEX encoding = UTF-8 Unicode
% !Mode:: "TeX:UTF-8"

\documentclass{resume}
\usepackage{zh_CN-Adobefonts_external} % Simplified Chinese Support using external fonts (./fonts/zh_CN-Adobe/)
% \usepackage{NotoSansSC_external}
% \usepackage{NotoSerifCJKsc_external}
% \usepackage{zh_CN-Adobefonts_internal} % Simplified Chinese Support using system fonts
\usepackage{linespacing_fix} % disable extra space before next section
\usepackage{cite}

\begin{document}
\pagenumbering{gobble} % suppress displaying page number

\name{毕成}

\basicInfo{
	\email{bc970321@g163.com} \textperiodcentered\ 
	\phone{(+86) 15765304024} \textperiodcentered\ 
	\github[bc970321]{https://github.com/bc970321}
}

\section{\faGraduationCap\ 教育背景}
\datedsubsection{\textbf{佳木斯大学}, 佳木斯}{2016 -- 至今}
\textit{在读本科生}\ 计算机科学与技术, 预计 2020 年 7 月毕业

\section{\faUsers\ 项目经历}


\datedsubsection{\textbf{分布式服务器运行状态监控系统}}{2019年3月 -- 2019年5月}
\role{C, Linux}{个人项目}
\begin{onehalfspacing}
	\begin{itemize}
		\item 项目简介:
		\\~
		针对服务器各个运行状态数据加以监控的简单监控软件。
		\item 运行状态监控数据采集:
		\\~
		通过 Linux 脚本获取服务器运行状态信息数据,并加以分析,分类存储。
		\item 数据及时性保证:
		\\~
		通过避免使用文件互斥锁来尽可能的来保证数据采集的及时性。
		\item 避免文件访问冲突方式:
		\\~
		通过使用短数据密集传输的方式来避免访问文件的文件指针冲突。
		\item 数据分类发送与接收方式:
		\\~
		通过发送标识码模拟请求,从而达到准确接收不同类型数据的目的。
		\item 数据传送可靠性保证:
		\\~
		数据传输采用 TCP 连接,以保证数据传输的可靠性。
	\end{itemize}
\end{onehalfspacing}

\datedsubsection{\textbf{文本快速查找}}{2019年1月 -- 2019年2月}
\role{Linux}{个人项目}
\begin{onehalfspacing}
	\begin{itemize}
		\item 项目简介:
		\\~
		基于字典树的简单多模文本查找工具。
		\item 字符编码:
		\\~
		可选择获取字符的二进制编码和哈弗曼编码。
		\item 字典树的建立:
		\\~
		线索化、建立失败指针、每个节点的子树由26颗改为2颗。
	\end{itemize}
\end{onehalfspacing}

% Reference Test
%\datedsubsection{\textbf{Paper Title\cite{zaharia2012resilient}}}{May. 2015}
%An xxx optimized for xxx\cite{verma2015large}
%\begin{itemize}
%  \item main contribution
%\end{itemize}

\section{\faCogs\ IT 技能}
% increase linespacing [parsep=0.5ex]
\begin{itemize}[parsep=0.5ex]
	\item 开发平台: 熟悉linux平台开发环境,拥有linux平台开发经验。
	\item 编程语言: 熟悉C,较熟悉c++、Python、Bash。
	\item 数据结构: 基本掌握顺序表、链表、栈、队列、字典树、AVL树、红黑树等数据结构。
	\item 基础算法: 熟悉简单查找算法(二分查找、三分查找)、排序算法(快速排序、插入排序等)、字符串匹配算法(AC自动机、ShiftAnd等)等基础算法。
\end{itemize}

\section{\faInfo\ 其他}
% increase linespacing [parsep=0.5ex]
\begin{itemize}[parsep=0.5ex]
  \item GitHub:  https://github.com/bc970321
  \item 校 ACM 实验室负责人。
  \item 第十四届黑龙江省大学生程序设计竞赛三等奖
\end{itemize}

%% Reference
%\newpage
%\bibliographystyle{IEEETran}
%\bibliography{mycite}
\end{document}
